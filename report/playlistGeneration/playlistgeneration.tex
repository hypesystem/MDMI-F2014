\section{Playlist Generation} % (fold)
\label{sec:playlist_generation}

This section details the \'practical\' application of our data mining efforts. First, the section will detail
the overall concept of creating a bridgin playlist between two songs. Next, we will describe the algorithms used in our applications. And finally, we will discuss the possibilities of the playlist.

\subsection{Concept}
By using an ESOM to cluster and project our data set, we can use the two-dimensional grid of neurons to create paths between songs. A \'bridging playlist\' is found by creating a path between two neurons on the grid and relating songs in our data set to neurons on the path. The path is the shortest path between two neurons with each neuron having a connection to its adjacent neurons. The playlist generation can be explained with the following steps:
\begin{itemize}
\item Select two songs from a list and identify their bestmatch neurons on the grid, \\
\item Let each neuron $ n_{k,i} $  have the property that it is connected to neurons $ [ n_{k+1,i} ; n_{k-1,n} ; n_{k,n+1} ; n_{k,n-1} ], $ \\
\item Create a shortest path from the previously selected bestmatch neurons, \\
\item Delete any neuron from the path that is not a bestmatch neuron, \\
\item Associate bestmatches with songs that constitute the playlist
\end{itemize}

This description of the concept presents two important challenges. First, how do we weigh each connection from neuron to neuron, and second, how do we associate bestmatch neurons with songs that qualify to fit the playlist?
These challenges are addressed next in the algorithms section.\\

\subsection{Algorithms}

To address the first challenge of connecting neurons and finding a path between them, we use a popular mode of abstraction: graph building and shortest path finding in graphs. We use a standard adjacency list based graph CITATION that represent nodes as vertices and connections between nodes as edge. Each edge is weighted by calculating the euclidean distance between the prototype vectors of each neuron. \\
Creating a path between two bestmatch neurons is then done by using Dijkstra\'s shortest path algorithm for non-negative weighted graphs CITATION. We will claim but not prove that if the graph is built properly, this algorithm guarantees the shortest path (if it exists) between two nodes. \\

The second problem of associating w  




% section playlist_generation (end)