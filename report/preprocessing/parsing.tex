\subsection{Parsing Data}
\label{sec:preprocessing_parsing}

HDF5, the fifth iteration of the \emph{hierachical data format}, is only supported by C, with a part of the API available
in C\# wrappers. These wrappers, however, are direct ports of the C API: not very object oriented, and not very easy to use.

To parse the data, an object oriented wrapper was created, handling the reading of data in a more accessible manner. HDF5 files
are organized in files, each file containing several groups (like folders in a normal file system). Each group contains
datasets of various types.

There are scalar and compound datasets, where a scalar dataset can be seen as a single multi-dimensional point, and compound
datasets are much like those found in relational databases (albeit without the relations): attribute names are used to describe
the different columns; each column has a specific type; and the dataset can have several rows.

\begin{figure}[h!]
	\begin{tabular}{|p{\textwidth}|}
		\hline
		\begin{verbatim}
			var file = new File(filename);
			var metadata_group = file.GetGroup("metadata");
			var song_data = metadata_group.GetCompoundDataset("songs");
			string artist_name = song_data[0].GetString("artist_name");
		\end{verbatim} \\
		\hline
	\end{tabular}
	\caption{Example of syntax built to read HDF5 files (found in the \emph{HDF5Reader} library).}
\end{figure}

The HDF5Reader reads to a virtual representation of songs (the Song class), which contains the all the relevant fields for our data
mining.

The LRN format is simply a tab-separated format, with the first few lines describing the contents of the file. Writing to this format
is straight-forward. In order to be able to load the data quickly for further data mining, we have created an LRNReader, which reads
the data to Songs.
