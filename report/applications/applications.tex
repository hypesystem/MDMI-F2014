\section{Conclusion} % (fold)
\label{sec:applications}
The results of the our solution are not easily evaluated, because the concept of similarity between songs has an inherent subjective component. It may be that the songs found connecting two specific songs is indeed the shortest path in terms of euclidean distance, but this has nothing to say about a person's psychoacoustic experience of the transitions between them. However, we have tried to evaluate the playlist statistically, as well as by listening to the play-lists and have found that the songs do indeed seem similar. Based on this, we conclude that the solution could be made sufficiently robust, perhaps given more time for experimenting with different attribute selection and different choices for edge-weights in the shortest path search.
\\\\
It is possible to imagine a number of applications of play-list generation of the type described in this report. For example it could be used to offer streaming service for an audience wanting a radio-station without sudden shifts in song choice. Similar services could be imagined for bars or restaurants.
\\\\
Software for DJs is another area of possible application. One could imagine a software solution that could analyse a DJ's music library, and then the DJ could use this analysis to search for paths between songs while performing. In this context, a possible extension to the solution could be customization of parameters for SOM training, so that some attributes deemed unimportant in some specific context, e.g when mixing a techno set versus a hip-hop set, could be excluded from the analysis. This would be an interesting continuation of the project.

% section applications (end)